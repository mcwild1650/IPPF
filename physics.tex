\documentclass{article}

\newcommand{\pd}{\partial}

\begin{document}

This program solves the Vorticity-Streamline equations for the flow of an incompressible fluid around a conic parabola at various modified Reynolds numbers and Circulation parameters with or without a synthetic jet modification.

The two dimensional, viscous, incompressible and unsteady Navier-Stokes equatioins for the inner flow around a canonic parabola are:

\[
\frac{\pd u}{\pd x} + \frac{\pd v}{\pd y} = 0
\]
\[
\frac{\pd u}{\pd t} + u \frac{\pd u}{\pd x} + v \frac{\pd u}{\pd y} = - \frac{\pd p}{\pd x} + \frac{1}{Re_M} \nabla^2 u
\]
\begin{equation}
\label{eq:NS1}
\frac{\pd v}{\pd t} + v \frac{\pd v}{\pd x} + v \frac{\pd v}{\pd y} = - \frac{\pd p}{\pd y} + \frac{1}{Re_M} \nabla^2 v
\end{equation}

Using

\begin{equation}
u=\frac{\pd \psi}{\pd y},\: v=\frac{\pd \psi}{\pd x},\: \mathrm{and}\: \omega=\frac{\pd v}{\pd x} - \frac{\pd u}{\pd y},
\end{equation}

(\ref{eq:NS1}) can be reduced to:

\[
\frac{\pd \omega}{\pd t} + \frac{\pd \psi}{\pd y} \frac{\pd \omega}{\pd x} - \frac{\pd \psi}{\pd x} \frac{\pd \omega}{\pd y} = \frac{1}{Re_M} \nabla^2\omega
\]
\begin{equation}
\frac{\pd^2 \psi}{\pd x^2} + \frac{\pd^2 \psi}{\pd y^2} = -\omega.
\label{eq:NS2D}
\end{equation}

Parabolic coordinates, $x=(\mu^2-\eta^2)/2$ and $y=\mu\eta$ are used to transform the field into a Cartesian coordinate space. Here $\mu$ is the coordinate parallel to the surface of the parabola and $\eta$  being the coordinate normal to the surface. The canonic parabola is now described by the $\eta=1$ surface and flow evolves in the domain $-\infty < \mu < \infty$, $\eta> 1$. In parabolic coordinates the velocity components can be given by:

\[
V_\mu=\frac{1}{\sqrt{\mu^2+\eta^2}} \frac{\pd \psi}{\pd \eta}, \quad\quad
V_\eta=\frac{-1}{\sqrt{\mu^2+\eta^2}} \frac{\pd \psi}{\pd \mu}
\]


Then, the Navier-Stokes equations (\ref{eq:NS2D}) become:
\begin{equation}
\frac{\pd \omega}{\pd t} + \frac{1}{\mu^2 + \eta^2} \left(\frac{\pd \psi}{\pd \eta} \frac{\pd \omega}{\pd \mu} - \frac{\pd \psi}{\pd \mu} \frac{\pd \omega}{\pd \eta} \right) = \frac{1}{Re_M} \frac{1}{\mu^2 + \eta^2} \left(\frac{\pd^2 \omega}{\pd \mu^2} + \frac{\pd^2 \omega}{\pd \eta^2} \right)
\end{equation}

\begin{equation}
\omega = \frac{-1}{\mu^2 + \eta^2} \left(\frac{\pd^2 \psi}{\pd \mu^2} + \frac{\pd^2 \psi}{\pd \eta^2} \right)
\end{equation}

The vorticity transport and stream function equations were re-arranged to produce the following conservative form:

\[
\frac{\pd \omega}{\pd t} + \frac{1}{\mu^2 + \eta^2} \left[\frac{\pd}{\pd \mu} \left(\sqrt{\mu^2 + \eta^2} V_\eta \omega \right) \right] = \frac{1}{Re_M(\mu^2 + \eta^2)} \left(\frac{\pd^2 \omega}{\pd \mu^2} + \frac{\pd^2 \omega}{\pd \eta^2} \right)
\]

\begin{equation}
\omega = \frac{-1}{\mu^2 + \eta^2} \left(\frac{\pd^2 \psi}{\pd \mu^2} + \frac{\pd^2 \psi}{\pd \eta^2} \right)
\label{eq:consv}
\end{equation}

Equations \ref{eq:consv} are subjected to the tangency and no slip flow conditions on the canonic parabola surface, i.e. $V_\eta(\mu,\eta=1) = 0$, $\psi(\mu,\eta=1) = 0$. Using the inversion of the coordinates transform, $\eta = \sqrt{ \sqrt{x^2+y^2} - x}$ and $\mu = sqrt{ \sqrt{x^2+y^2} + x}$, it can be shown that the far-field behavior as described by (A.5) results in
$\Phi^* \approx \frac{1}{2} (\mu^2+\eta^2) + \eta * \tilde{A}\mu$.
In the far field:
$V_\mu = \frac{\mu + \tilde{A}}{\sqrt{\mu^2 + \eta^2}}$ and $V_\eta = \frac{1 - \eta}{\sqrt{\mu^2 + \eta^2}}$, from which:

\begin{equation}
\psi = \mu\eta - \mu + \tilde{A}\eta
\end{equation}
%
as both $\mu$ and $\eta$ tend to infinity. Note that the far field behavior, when applied to the whole domain, is the inviscid, steady-state solution of the problem for all $\tilde{A}$ .

For a numerical implementation, the semi-infinite domain $\eta > 1$ is reduced to the domain $-\mu_{max} < \mu < \mu_{max}$ and $1 < \eta < \eta_{max}$, where $\mu_{max}$ and $\eta_{max}$ are sufficiently large. This domain is discretized by a uniform mesh with constant step sizes in both directions $\Delta \mu$ and $\Delta \eta$ respectively. The index of each grid point is ($i$,$j$) respectively, where $-M \le i \le M$, $1 \le j \le N$. Time is discretized by constant time steps $\Delta t$ with index $n$ for each time level. The time
derivative in
(\ref{eq:consv})
is approximated by a first-order forward difference and second-order central
differences are used to approximate the spatial derivatives. The discretized formulation of (\ref{eq:consv}) is:

%(9)

%(10)

Equation (\ref{}) is solved by the Jacobi iteration method. Once convergence to a given tolerance is achieved, the velocity field at time level $n+1$ can be determined from:

\begin{equation}
V_\mu^{n+1}=\frac{1}{\sqrt{\mu_{i,j}^2+\eta_{i,j}^2}} \frac{\psi_{i,j+1}^{n+1} - \psi_{i,j-1}^{n+1}}{2\Delta \eta}, \quad V_\eta^{n+1}=\frac{-1}{\sqrt{\mu_{i,j}^2+\eta_{i,j}^2}} \frac{\psi_{i+1,j}^{n+1} - \psi_{i-1,j}^{n+1}}{2\Delta \mu}
\end{equation}
%(11)
%Equations (9) and (10) are solved under the following conditions: (1) a wall boundary
%condition  in, j 1  0 for all  M  i  M ; (2) an inflow far-field  in, j  N , given by the far-field
%potential flow behavior (8) and in, j  N  0 are employed for  M  i  M ; (3) an outflow
% in  M , j , given by the far-field flow behavior (8) and in  M , j  0 are used for N BL  j  N ; (4) a
% 
%Neumann boundary condition 
%(BCS)
%along the outflow boundaries
%i   M for 1  j  N BL   which allows the outflow to evolve naturally. Here N BL  N / 20 is used; 
%(5) the vorticity in, j 1 (along the parabola surface j  1 ) for  M  i  M is computed by a
%second order, forward difference approximation in the j direction which also accounts for the
%wall no-slip condition along a stationary boundary (see details in Hoffmann & Chiang 1993), i.e.
%in, j 1  7 in, j 1  8 in, j  2   in, j  3 / 2( y ) 2 .
%The computations are first-order accurate in time and second-order accurate in space and are
%consistent with the original equations (7) as the mesh is refined. The field of velocity and
%12
%Page 13 of 50
%vorticity at time level n+1 are used to advance the solution to the next time step. For a given
%~
%ReM and A , the solution of (9) through (11) is advanced in time until time-asymptotic behavior,
%steady or periodic, is achieved.
%~
%The computations are initiated in the following way. At a given ReM and A  0 we start with
%the inviscid, potential flow solution as an initial state and march in time until a steady, time-
%asymmtotic state of the viscous flow problem is found. Then, we use this solution as an initial
%state for the computation of the flow evolution at same ReM at an increased incremental value
%~
%~
%of A , for example A  0.1. Similarly, each time-asymptotic state is used as an initial state for the
%~
%computation of flow evolution at the next nearby value of A.
%Certain numerical stability criteria must be satisfied in the computations. Here we require
%that the Courant number Ck, diffusion number dk, and the cell Reynolds number ReC:
%(12)
%obey certain limitations. In the present numerical calculations, a=Umax=1 and the subscript k on the
%spatial difference indicates or . Extending von Neumann numerical linear stability analysis (see
%Hoffmann & Chiang, 1993, chapter 4 and Roache, 1998 chapter 3) to the present forward in time-
%central in space differencing scheme leads to the stability requirements in a two- dimensional
%problem: C = C+C ≤ 1, d = d+d ≤ 1/2, and ReC ≤ 4/C. For example: with ReM = 700 and 
%and  are 0.2 and 0.025, respectively, the Courant number dictates that t be less than 0.0222 and
%the diffusion number states it must be less than 0.215 while the cell Reynolds number (~141 in this
%case) criterion calls for a t ≤ 0.00063. Clearly, for this case, the cell Reynolds number criterion is
%the most restrictive and is therefore used as the maximum threshold for t, dictating small time steps.
%Further, we refer to the work by Thompson, Webb & Hoffman (1985), later verified by Sousa
%(2003), which states that the cell Reynolds number restriction is overly restrictive for stable
%calculations. In spite of this fact, we proceed to use it as a buffer against numerical instabilities that
%may result from non-linear effects. As a result, the CFL number is less than 0.01 which provides a
%high accuracy of resolution of velocity signals in time, specifically of the low-frequency T-S waves
%that convect along the parabola surface and are involved in the delay of stall.




\end{document}
