\documentclass[twocolumn]{article}
\usepackage{graphicx}
\usepackage{amsmath}
\newcommand{\myimage}[3]{
\begin{figure}[!ht]
\caption{#2}
\label{#3}
\includegraphics[width=0.4\textwidth]{#1}
\end{figure}
}
\title{Interactive Parallel Parabola Flow}
\author{
Micah Corah
\and
Dan Ibanez
\and
David McWilliams
\and
Han Wang
}
\begin{document}
\maketitle
\section{Introduction}
\section{Physics}
%David
\section{Parallel Calculation and Optimization}
%Han
\section{Parallel Boundary Conditions}
%Micah
\section{Parallel File IO}
%Dan
In order to easily verify our results using the serial code,
we chose to implement single-file IO.
On the other hand, we would like to get some parallelism in
file IO so that it does not cripple scaling, so we try to use
MPI's parallel file IO routines to write the file in parallel.
An additional complication is that the file is in a text format
rather than binary.
Fortunately, the file is formatted such that all values are
padded to a known character width, so the offset in characters
of every value and the total file size are easily computable.
To write our output file, we first open a file collectively
with \texttt{MPI\_File\_open} and then 
allocate enough characters for the whole output with
\texttt{MPI\_File\_preallocate}.
We use
\texttt{sprintf} to produce a text string of know length for
each floating-point value, and then use
\texttt{MPI\_File\_write\_at} to write the text string at
the correct location in the file.
Each process computes its own offsets and writes concurrently
without any synchronization until all processes close the file.
This is theoretically a scalable algorithm, although in practice
operating system and MPI limits are likely to limit scaling
to some upper bound.
\section{Results}
\section{Future Work}
\section{Conclusions}
\section{References}

\end{document}
