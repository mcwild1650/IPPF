\documentclass[twocolumn]{article}
\usepackage{graphicx}
\usepackage{amsmath}
\usepackage{verbatim}
\usepackage{paralist}
\newcommand{\myimage}[3]{
\begin{figure}[!ht]
\caption{#2}
\label{#3}
\includegraphics[width=0.4\textwidth]{#1}
\end{figure}
}
\newcommand{\pd}{\partial}
\title{Interactive Parallel Parabola Flow}
\author{
Micah Corah
\and
Dan Ibanez
\and
David McWilliams
\and
Han Wang
}
\begin{document}
\maketitle
\section{Introduction}

We present the parallelization of a finite difference fluid dynamics
program using MPI and scaling results on the CCNI IBM
Blue Gene/Q.
The program implements a solution of 2D flow over a structured grid.
It maintains several fields over this grid which are output at
the end of the simulation.
Due to the very local finite difference stencil, the use of explicit
integration, and the structured rectangular grid,
this program provides one logical approach to distributed-memory
parallelism.
By partitioning the grid computation
and file I/O among processes, we show
good strong scaling up to 4096 processes on the Blue Gene/Q.
This report actually begins with Section \ref{sec:physics},
describing the physical equations and simulation purpose,
which is found near the end for formatting purposes.

\section{Parallel Operation and Optimization}

The whole calculation of this code is a iteration procedure.
Matrix Psi is the matrix of streamlines and matrix Omega is the matrix of vortices.
Matrix Psi is calculated in function PsiCalc with an iterative method.
Function PsiCalc is the most time consuming function in the serial version
of this code.
It may need several or tens of iterations to finish.
As a finite difference problem, new Psi and Omega matrices are calculated from
the old Psi and Omega matrices.
Every elements in new Psi and omega matrix is a function of 5 elements in the old
matrix.
\[\Psi_{i,j}=f(\Psi^0_{i,j},\Psi^0_{i-1,j},\Psi^0_{i+1,j},\Psi^0_{i,j+1},\Psi^0_{i,j-1})\]
%\[New\_Psi[i][j]=
%Function( Old\_Psi[i][j],
%          Old\_Psi[i-1][j],
%          Old\_Psi[i+1][j],
%          Old\_Psi[i][j+1],
%          Old\_Psi[i][j-1] )\]
For the parallelization scheme, a processor only stores and calculates a part
of the Psi and Omega matrix.
The Psi and omega are all (Nx+2)*(My+2) matrix.
They both have two rows at the top and bottom and two columns at the right
and left that considered as boundary, which are set in the BCs function.
We use a highly symmetric configuration for parallelization.
Every processors calculate and store a (nx+2)*(my+2) matrix.
They also have  two rows at the top and bottom and two columns at the right
and left that are considered as boundary.
And they do not calculate the boundary.
Small matrices  overlap with the adjacent matrix with two rows or two columns.
The boundary of each processors are calculated by other processors.

\myimage{configuration.png}{Configuration of Processors}{fig:configuration}

Figure  \ref{fig:configuration} is an example.
In figure \ref{fig:configuration}, "o" is an element in the Psi or Omega matrix.
Row 1,6 and column 1,8 are the boundaries.
Here, we have 4 processors.
Processor 1 stores the part covered by red square
which ranges from row 1 to 5 and column 1 to 5.
Processor 2 stores the part covered by green square,which ranges from
row 1 to 5 and from column 4 to 8.
Processor 3 stores the elements covered by the blue square.
But processor 1 only calculate elements from row 2 to row 4 and from column
2 to column 4.
The rest elements are considered as the boundary of processor 1.
Although column 5 is the boundary of processor 1,  it is not the boundary
of processor 2.
So processor 2 will calculate column 5 and send the value of these elements
to processor 1. In the same way, processor 3 will calculate row 5 and send
them to processor 1.
So processor 1 will have updated value for all the elements.
For the same reason, all the processors will get the new values for the matrices
after one iteration.
The criterion to stop the iteration is the maximum difference between
the old matrix and the new matrix is smaller than a specific value PsiTol.
So when finishing an iteration, the maximum difference between the old Psi
matrix and the new Psi matrix is calculated in all the processor.
We use \texttt{MPI\_Allreduce} to gather the maximum value in different
processors and compare them with PsiTol to determine whether to continue
next iteration or not.

\section{Input}

The serial program uses a command-line dialog to query the user for
configuration parameters before starting a simulation.
This includes grid size, fluid physical parameters, simulation
time and timestep, and artificial jet specifications, if any.

In parallel, such information should not be queried through the
command line, especially for a batch job.
We developed a configuration file format to store this information.
In parallel, our program uses rank 0 to read the configuration file
and broadcast the data to other ranks.
Editing this file is also a convenient way to change parameters compared
with re-running the command line dialog.
We retain the dialog, which is accessible using the ``config" argument
when running the program, since it is a descriptive way to generate
an initial configuration.

Once a configuration is generated, a simulation can be started
by giving the program two arguments: ``run" and the name of the
configuration file.

\section{Partitioning}

The parallel operation of this program requires us to partition
the structured rectangular grid on which fields are defined.
We use a rather naive and expensive but effective partitioner.
It tries all possible partitions of a particular class
and chooses the best one.
In this case a partition is better than another partition
if it has both less maximum computation and less
total communication.
Maximum computation is the largest number of grid points given to
any process, and communication is the total number of grid points
that have to be exchanged to synchronize a field after on step
of computation.

The class of partitions we choose are those for which two integers
$p_x,p_y$ exist such that the number of processes is $p_xp_y=p$.
Each rank is assigned coordinates $(i,j)\in p_x\times p_y$.
Most ranks ($i<p_x,j<p_y$), receive a sub-grid of the global
$m\times n$ grid which is of size
$\lceil\frac{m}{p_x}\rceil\times \lceil\frac{n}{p_y}\rceil$.
The other ranks receive the remaining grid points along the top
and right of the grid.
This class of partitions has some benefits and limitations.
The main benefit is that maximum workload is well minimized.
Most ranks have the maximum workload, and only a few have less work.
The limitation is in the division process along each dimension,
e.g. not all values of $m$ and $p_x$ are valid.
In that case that all possible values of $p_x,p_y$ are incompatible with
the grid dimensions $m,n$, the partitioner fails.
This did, in fact, occur in some of our experiments.

\section{Parallel Boundary Conditions}

For any interior point in either Psi or Omega, the value is calculated
as a function of values in matrix cells one step to the north, south, east, or
west. This is not true of the extreme boundaries of the matrix. Therefore,
calculation of boundaries requires use of both the local and global position
within the matrix. This results in greater deviation in code structure from
the serial version when calculating the boundary conditions than the interior.
For any local boundaries, a check is performed to determine whether the
local boundary is also a global boundary. For many of the boundaries
that is the only changes necessary, the value being calculated in a manner
with similar requirements to the interior. Jets and boundary layers each
introduce requirements for global position in the boundary layer.
The side boundaries involve boundary layers which depend additionally
on the side of the boundary layer where particular cell exists. Jets affect
the lower boundary, and calculations vary by whether a cell is located
before, within, or after the region of the jet. Within the jet, Psi
is determined using just the simulation time and the linear position
of the cell. However, past the end of the jet, each cell on the Psi
boundary is always equal to the value of the last cell within the jet.
This introduces a problem when this last cell is not within the local
matrix. Generally, this kind of issue is dealt with by having the
owning process broadcast the value of the cell. Fortunately, values
within the jet do not depend on the values of any fields. The only
value required that is not stored locally is simply the linear position
which would have been stored in the position vector. This value can
be quickly calculated, and, as a result, the value for the jets can be
computed once locally without requiring any additional communication.
The only additional concern is that the lower boundary of Omega requires
values of Psi two steps into the field instead of one. This does not
actually introduce any new requirements. For this calculation to fail,
the local dimension in $Y$ would be one. Such a partitioning should
never occur as every owned row would have to be sent twice per round
of computation.

\section{Parallel File IO}

In order to easily verify our results using the serial code,
we chose to implement single-file IO.
On the other hand, we would like to get some parallelism in
file IO so that it does not cripple scaling, so we try to use
MPI's parallel file IO routines to write the file in parallel.
An additional complication is that the file is in a text format
rather than binary.
Fortunately, the file is formatted such that all values are
padded to a known character width, so the offset in characters
of every value and the total file size are easily computable.
To write our output file, we first open a file collectively
with \texttt{MPI\_File\_open} and then 
allocate enough characters for the whole output with
\texttt{MPI\_File\_preallocate}.
We use
\texttt{sprintf} to produce a text string of know length for
each floating-point value, and then use
\texttt{MPI\_File\_write\_at} to write the text string at
the correct location in the file.
Each process computes its own offsets and writes concurrently
without any synchronization until all processes close the file.
Since there is no overlap or contention in the regions being
written to,
this is theoretically a scalable algorithm.
In practice, we did worry about
operating system and MPI limits related to only using one file,
but the results show otherwise.

\section{Results}

Tests on the BlueGene/Q were performed using two configurations.
The smaller configuration file has dimensions of $800\times 1600$,
and runs for 10000 time steps:\\
\verbatiminput{mpi/bigtestconfig.txt}
This is a test of a size that can be reasonably computed in serial
and serves as an example of how IPPF provides faster run times when
compared to tests that would have been performed serially.
The larger configuration file has dimensions of $3200\times 6400$
and runs for only 1000 time steps. This an example of a larger
test more likely to be comparable to runs intended to be in parallel.
\verbatiminput{mpi/Bigtestconfig.txt}

Using both these configurations, we perform strong scaling studies
using 1,2, and 4 processes per core, always starting with
a single node with all cores utilized.
All the following figures show the runtime of some portion
of the simulation plotted over process count for several ratios
of processes per core.
The two sections we profiled were the field computation for all
timesteps and the file writing time.
The first thing to note is that both these critical sections exhibit
essentially perfect linear strong scaling for all cases run.
In particular, the MPI File IO implementation was able to concurrently
write to one file from thousands of processes.

Figure \ref{fig:calc1} shows the time for calculation using the
$800\times 1600$ case.
In this case, the results for 2 and 4 processes per core overlap,
indicating that there is a slight benefit to using 1 process per core
which is extinguished with any higher count.
Figure \ref{fig:file1} shows the runtime for writing restart files
in parallel for the first input.
Again 2 and 4 processes per core have the same result.
For the $3200\times 6400$ test case, Figures \ref{fig:calc2} and
\ref{fig:file2} show the runtime of field calculation and file IO,
respectively.
In both of these figures we note that there is a slight penalty
when moving from 1 to 2 processes per core, and a higher penalty
moving from 2 to 4.

\myimage{one_calc.png}{Field computation scaling for input 1}{fig:calc1}
\myimage{one_file.png}{File IO scaling for input 1}{fig:file1}
\myimage{two_calc.png}{Field computation scaling for input 2}{fig:calc2}
\myimage{two_file.png}{File IO scaling for input 2}{fig:file2}

\section{Source Code}

The file \texttt{driver.c} implements the high-level control of the program,
including selecting between a dialog interface or running from a configuration
file.
The \texttt{config.c} file implements configuration handling, including
the actual dialog, file IO for configuration files, and broadcast
of configuration data.
The configuration structure is kept contiguous for easy broadcasting.
The \texttt{grid.c} file implements the partitioning algorithm and
returns important variables such as processes along X and Y, local
process coordinates, etc.
The \texttt{space.c} file computes X and Y coordinates for each local
grid point.
A file \texttt{field.c} provides helper functions for dealing with
values defined on a 2D grid.
The \texttt{calc.c} file actually implements the majority of the computation,
including exchange of partition boundary field values and reductions
of field differences.
The \texttt{restart.c} file implements serial and parallel versions
of the output file IO, including the scalable MPI-based writing.
Finally, \texttt{tools.c} implements basic helper functions for
memory management, flow control, and communication.
The source code can be found under Dan's account (\texttt{PPCibaned})
on the Kratos machine.

\section{Conclusions}

We presented a straightforward parallelization of a finite difference
solver for 2D fluid equations around a parabolic airfoil front.
Using a 2D partitioning that minimizes maximum local computation
and communication, we achieve linear scaling up to 4096 processes.
In addition, we maintain almost exactly the same input and output
formats so that this program could be used directly by the same
community currently using the serial version.
Our general partitioner and portable MPI-based implementation allows the program
to be used on any machine which has MPI, not just the Blue Gene.
This program is expected to be used for real engineering tasks
in the future, and was kept modular and interactive for usability.

\onecolumn
\section{Physics}
\label{sec:physics}
\documentclass{article}

\newcommand{\pd}{\partial}

\begin{document}

This program solves the Vorticity-Streamline equations for the flow of an incompressible fluid around a conic parabola at various modified Reynolds numbers and Circulation parameters with or without a synthetic jet modification.

The two dimensional, viscous, incompressible and unsteady Navier-Stokes equatioins for the inner flow around a canonic parabola are:

\[
\frac{\pd u}{\pd x} + \frac{\pd v}{\pd y} = 0
\]
\[
\frac{\pd u}{\pd t} + u \frac{\pd u}{\pd x} + v \frac{\pd u}{\pd y} = - \frac{\pd p}{\pd x} + \frac{1}{Re_M} \nabla^2 u
\]
\begin{equation}
\label{eq:NS1}
\frac{\pd v}{\pd t} + v \frac{\pd v}{\pd x} + v \frac{\pd v}{\pd y} = - \frac{\pd p}{\pd y} + \frac{1}{Re_M} \nabla^2 v
\end{equation}

Using

\begin{equation}
u=\frac{\pd \psi}{\pd y},\: v=\frac{\pd \psi}{\pd x},\: \mathrm{and}\: \omega=\frac{\pd v}{\pd x} - \frac{\pd u}{\pd y},
\end{equation}

(\ref{eq:NS1}) can be reduced to:

\[
\frac{\pd \omega}{\pd t} + \frac{\pd \psi}{\pd y} \frac{\pd \omega}{\pd x} - \frac{\pd \psi}{\pd x} \frac{\pd \omega}{\pd y} = \frac{1}{Re_M} \nabla^2\omega
\]
\begin{equation}
\frac{\pd^2 \psi}{\pd x^2} + \frac{\pd^2 \psi}{\pd y^2} = -\omega.
\label{eq:NS2D}
\end{equation}

Parabolic coordinates, $x=(\mu^2-\eta^2)/2$ and $y=\mu\eta$ are used to transform the field into a Cartesian coordinate space. Here $\mu$ is the coordinate parallel to the surface of the parabola and $\eta$  being the coordinate normal to the surface. The canonic parabola is now described by the $\eta=1$ surface and flow evolves in the domain $-\infty < \mu < \infty$, $\eta> 1$. In parabolic coordinates the velocity components can be given by:

\[
V_\mu=\frac{1}{\sqrt{\mu^2+\eta^2}} \frac{\pd \psi}{\pd \eta}, \quad\quad
V_\eta=\frac{-1}{\sqrt{\mu^2+\eta^2}} \frac{\pd \psi}{\pd \mu}
\]


Then, the Navier-Stokes equations (\ref{eq:NS2D}) become:
\begin{equation}
\frac{\pd \omega}{\pd t} + \frac{1}{\mu^2 + \eta^2} \left(\frac{\pd \psi}{\pd \eta} \frac{\pd \omega}{\pd \mu} - \frac{\pd \psi}{\pd \mu} \frac{\pd \omega}{\pd \eta} \right) = \frac{1}{Re_M} \frac{1}{\mu^2 + \eta^2} \left(\frac{\pd^2 \omega}{\pd \mu^2} + \frac{\pd^2 \omega}{\pd \eta^2} \right)
\end{equation}

\begin{equation}
\omega = \frac{-1}{\mu^2 + \eta^2} \left(\frac{\pd^2 \psi}{\pd \mu^2} + \frac{\pd^2 \psi}{\pd \eta^2} \right)
\end{equation}

The vorticity transport and stream function equations were re-arranged to produce the following conservative form:

\[
\frac{\pd \omega}{\pd t} + \frac{1}{\mu^2 + \eta^2} \left[\frac{\pd}{\pd \mu} \left(\sqrt{\mu^2 + \eta^2} V_\eta \omega \right) \right] = \frac{1}{Re_M(\mu^2 + \eta^2)} \left(\frac{\pd^2 \omega}{\pd \mu^2} + \frac{\pd^2 \omega}{\pd \eta^2} \right)
\]

\begin{equation}
\omega = \frac{-1}{\mu^2 + \eta^2} \left(\frac{\pd^2 \psi}{\pd \mu^2} + \frac{\pd^2 \psi}{\pd \eta^2} \right)
\label{eq:consv}
\end{equation}

Equations \ref{eq:consv} are subjected to the tangency and no slip flow conditions on the canonic parabola surface, i.e. $V_\eta(\mu,\eta=1) = 0$, $\psi(\mu,\eta=1) = 0$. Using the inversion of the coordinates transform, $\eta = \sqrt{ \sqrt{x^2+y^2} - x}$ and $\mu = sqrt{ \sqrt{x^2+y^2} + x}$, it can be shown that the far-field behavior as described by (A.5) results in
$\Phi^* \approx \frac{1}{2} (\mu^2+\eta^2) + \eta * \tilde{A}\mu$.
In the far field:
$V_\mu = \frac{\mu + \tilde{A}}{\sqrt{\mu^2 + \eta^2}}$ and $V_\eta = \frac{1 - \eta}{\sqrt{\mu^2 + \eta^2}}$, from which:

\begin{equation}
\psi = \mu\eta - \mu + \tilde{A}\eta
\end{equation}
%
as both $\mu$ and $\eta$ tend to infinity. Note that the far field behavior, when applied to the whole domain, is the inviscid, steady-state solution of the problem for all $\tilde{A}$ .

For a numerical implementation, the semi-infinite domain $\eta > 1$ is reduced to the domain $-\mu_{max} < \mu < \mu_{max}$ and $1 < \eta < \eta_{max}$, where $\mu_{max}$ and $\eta_{max}$ are sufficiently large. This domain is discretized by a uniform mesh with constant step sizes in both directions $\Delta \mu$ and $\Delta \eta$ respectively. The index of each grid point is ($i$,$j$) respectively, where $-M \le i \le M$, $1 \le j \le N$. Time is discretized by constant time steps $\Delta t$ with index $n$ for each time level. The time
derivative in
(\ref{eq:consv})
is approximated by a first-order forward difference and second-order central
differences are used to approximate the spatial derivatives. The discretized formulation of (\ref{eq:consv}) is:

%(9)

%(10)

Equation (\ref{}) is solved by the Jacobi iteration method. Once convergence to a given tolerance is achieved, the velocity field at time level $n+1$ can be determined from:

\begin{equation}
V_\mu^{n+1}=\frac{1}{\sqrt{\mu_{i,j}^2+\eta_{i,j}^2}} \frac{\psi_{i,j+1}^{n+1} - \psi_{i,j-1}^{n+1}}{2\Delta \eta}, \quad V_\eta^{n+1}=\frac{-1}{\sqrt{\mu_{i,j}^2+\eta_{i,j}^2}} \frac{\psi_{i+1,j}^{n+1} - \psi_{i-1,j}^{n+1}}{2\Delta \mu}
\end{equation}
%(11)
%Equations (9) and (10) are solved under the following conditions: (1) a wall boundary
%condition  in, j 1  0 for all  M  i  M ; (2) an inflow far-field  in, j  N , given by the far-field
%potential flow behavior (8) and in, j  N  0 are employed for  M  i  M ; (3) an outflow
% in  M , j , given by the far-field flow behavior (8) and in  M , j  0 are used for N BL  j  N ; (4) a
% 
%Neumann boundary condition 
%(BCS)
%along the outflow boundaries
%i   M for 1  j  N BL   which allows the outflow to evolve naturally. Here N BL  N / 20 is used; 
%(5) the vorticity in, j 1 (along the parabola surface j  1 ) for  M  i  M is computed by a
%second order, forward difference approximation in the j direction which also accounts for the
%wall no-slip condition along a stationary boundary (see details in Hoffmann & Chiang 1993), i.e.
%in, j 1  7 in, j 1  8 in, j  2   in, j  3 / 2( y ) 2 .
%The computations are first-order accurate in time and second-order accurate in space and are
%consistent with the original equations (7) as the mesh is refined. The field of velocity and
%12
%Page 13 of 50
%vorticity at time level n+1 are used to advance the solution to the next time step. For a given
%~
%ReM and A , the solution of (9) through (11) is advanced in time until time-asymptotic behavior,
%steady or periodic, is achieved.
%~
%The computations are initiated in the following way. At a given ReM and A  0 we start with
%the inviscid, potential flow solution as an initial state and march in time until a steady, time-
%asymmtotic state of the viscous flow problem is found. Then, we use this solution as an initial
%state for the computation of the flow evolution at same ReM at an increased incremental value
%~
%~
%of A , for example A  0.1. Similarly, each time-asymptotic state is used as an initial state for the
%~
%computation of flow evolution at the next nearby value of A.
%Certain numerical stability criteria must be satisfied in the computations. Here we require
%that the Courant number Ck, diffusion number dk, and the cell Reynolds number ReC:
%(12)
%obey certain limitations. In the present numerical calculations, a=Umax=1 and the subscript k on the
%spatial difference indicates or . Extending von Neumann numerical linear stability analysis (see
%Hoffmann & Chiang, 1993, chapter 4 and Roache, 1998 chapter 3) to the present forward in time-
%central in space differencing scheme leads to the stability requirements in a two- dimensional
%problem: C = C+C ≤ 1, d = d+d ≤ 1/2, and ReC ≤ 4/C. For example: with ReM = 700 and 
%and  are 0.2 and 0.025, respectively, the Courant number dictates that t be less than 0.0222 and
%the diffusion number states it must be less than 0.215 while the cell Reynolds number (~141 in this
%case) criterion calls for a t ≤ 0.00063. Clearly, for this case, the cell Reynolds number criterion is
%the most restrictive and is therefore used as the maximum threshold for t, dictating small time steps.
%Further, we refer to the work by Thompson, Webb & Hoffman (1985), later verified by Sousa
%(2003), which states that the cell Reynolds number restriction is overly restrictive for stable
%calculations. In spite of this fact, we proceed to use it as a buffer against numerical instabilities that
%may result from non-linear effects. As a result, the CFL number is less than 0.01 which provides a
%high accuracy of resolution of velocity signals in time, specifically of the low-frequency T-S waves
%that convect along the parabola surface and are involved in the delay of stall.




\end{document}


\end{document}
